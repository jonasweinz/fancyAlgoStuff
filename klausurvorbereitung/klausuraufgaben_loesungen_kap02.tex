% header
\documentclass[10pt,a4paper]{article}

\usepackage[utf8]{inputenc}
\usepackage{hyperref}
\usepackage{amssymb}
\usepackage{amsthm}
\usepackage{amsmath}
\usepackage{ngerman}
\usepackage{gensymb}
\usepackage{enumitem}
\usepackage{stmaryrd}
\usepackage{listings}


%Das brauchen wir:
\newcommand{\zz}{\mathrm{Z\kern-.3em\raise-0.5ex\hbox{Z}}}

% the document
\begin{document}

% create the title
\title{Klausuraufgaben \\
\small{Kapitel 8 - Algorithmen auf Graphen}}
\author{Tom Kneiphof}
\date{\today}
\maketitle

\section*{Aufgabe 1}

\paragraph{a)}
Ein Matching $M \subseteq E$ eines ungerichteten Graphens $G = (V, E)$ ist eine Teilmenge der Kanten, so dass kein Knoten an zwei Kanten beteiligt ist.
\begin{equation*}
\forall \{v,w\}, \{x, y\} \in M \subseteq E : \{v, w\} \cap \{x, y\} = \emptyset
\end{equation*}


\paragraph{b)}
Bei einem \textbf{Nicht erweiterbaren Matching} handelt es sich um ein Matching $M$, zu dem keine Kante aus $E$ hinzugenommen werden kann, so dass es ein Matching bleibt.
\begin{equation*}
\forall \{v, w\} \in E \setminus M : M \cup \{ \; \{v, w\} \; \} \; \textnormal{ist kein Matching}
\end{equation*}

Bei einem \textbf{Maximales Matching} handelt es sich um ein Matching $M_{max}$ maximaler Gr\"o{\ss}e.
Es gibt also kein Matching welches mehr Kanten enth\"alt als $M_{max}$. Es gilt:
\begin{equation*}
\forall M : | M | \leq | M_{max} |
\end{equation*}


\paragraph{c)}
Satz von Berge: Ein Matching $M$ ist genau dann maximal, falls es keinen $M$-augmentierenden Pfad gibt.

\paragraph{d)}
Beweis des Satzes von Berge:

\paragraph{$\Rightarrow$}
Annahme: Es existiert ein $M$-augmentierender Pfad $P$.
Definiere $M' := M \oplus P$. Dann gilt $|M'| > |M|$, da $n$ Kanten entfernt wurden und $n+1$ Kanten hinzugef\"ugt wurden.
Dies ist ein Widerspruch dazu, dass $M$ bereits ein Maximales Matching war.

\paragraph{$\Leftarrow$}
Sei $M$ ein Matching in $G$, und existiere kein $M$-augmentierender Pfad. Sei weiter $M_{max}$ ein beliebiges maximales Matching in $G$. Annahme: $|M| < |M_{max}|$.
Betrachte $G' := (V, M \oplus M_{max})$. Da aus beiden Matchings je h\"ochstens eine Kante inzident zu $v$ sein kann, gilt:
\begin{equation*}
\forall v \in V : deg_{G'}(v) \leq 2
\end{equation*}
Die Zusammenhangskomponenten von $G'$ sind also isolierte Knoten, Kreise gerader L\"ange mit abwechselnden Kanten aus $M$ und $M_{max}$ oder Pfade mit abwechselnden Kanten aus $M$ und $M_{max}$.
Da $|M| < |M_{max}|$ angenommen wurde, gibt es einen $M$-augmentierenden Pfad $P$ in $G'$, dessen Endknoten beide $M$-frei sind. Da $P$ auch in $G$ existiert, gibt es einen $M$-augmentierenden Pfad in $G$, was ein Widerspruch dazu ist, dass ein solcher Pfad nicht existiert.

\section*{Aufgabe 2}
% Transformiere Nach erreichbarkeit in G_m
% + korrektheit
Gegeben sei ein Bipartiter Graph $G = (A, B, E)$ sowie ein Matching $M$ auf $G$. Konstruiere gerichteten Graphen $G_M = (A', B', E')$ wie folgt:
\begin{equation*}
A' := A \cup \{s\}  \qquad B':=B \cup \{t\} \qquad s,t \notin A \cup B \qquad s \neq t
\end{equation*}
Die Kanten die im Matching sind, werden von $A$ nach $B$ gerichtet und die Katen die nicht im Matching sind von $B$ nach $A$ gerichtet. Von $s$ aus sind die Knoten erreichbar die in $B$ $M$-frei sind, und von $M$-freien Knoten in $A$ aus ist $t$ erreichbar.
\begin{equation*}
E_1 := \{ \, (u, v) \mid (u, v) \in M, u \in A, v \in B \, \} \\
\end{equation*}
\begin{equation*}
E_2 := \{ \, (u, v) \mid (u, v) \in E \setminus M, u \in B, v \in A \, \}
\end{equation*}
\begin{equation*}
E_3 := \{ \, (s, b) \mid b \in B \textnormal{ ist $M$-frei} \, \} \cup \{ \, (a, t) \mid a \in A \textnormal{ ist $M$-frei} \, \}
\end{equation*}
\begin{equation*}
E' := E_1 \cup E_2 \cup E_3
\end{equation*}
Nun existiert in $G_M$ genau dann ein Pfad von $s$ nach $t$, wenn ein $M$-augmentierender Pfad in $G$ existiert.
\paragraph{Beweis}

\paragraph{$\Rightarrow$}
Sei $P = s, v_0, \ldots, v_k, t$ ein Pfad von $s$ nach $t$ in $G_M$.
Daraus folgt direkt, dass $v_0 \in B$ $M$-frei ist, sowie $v_k \in A$ $M$-frei.
Aus der Konstruktion von $E'$ folgt, dass die Kanten $(v_i, v_{i+1})$ abwechselnd aus $M$ und $E \setminus M$ kommen.
Also ist $v_0, \ldots, v_k$ ein $M$-augmentierender Pfad in $G$.

\paragraph{$\Leftarrow$}
Sei $P = v_0, \ldots, v_k$ ein $M$-augmentierender Pfad in $G$. O.B.d.A. seien $v_0 \in B$ und $v_k \in A$.
\begin{enumerate}
	\item $(s, v_0) \in E'$ da $v_0$ $M$-frei ist, und $v_0 \in B$
	\item $(v_k, t) \in E'$ da $v_k$ $M$-frei ist, und $v_k \in A$
	\item $v_i \in A$ falls $i$ ungerade
	\item $v_i \in B$ falls $i$ gerade
	\item $\{v_i, v_{i+1}\} \in M$ falls $i$ ungerade
	\item $\{v_i, v_{i+1}\} \in E \setminus M$ falls $i$ gerade
\end{enumerate}
Es existiert also eine Kante $(s, v_0) \in E'$ und $(v_k, t) \in E'$,
sowie die Kanten $(v_i, v_{i+1}) \in E'$.
Damit ist $P' = s, v_0, \ldots, v_k, t $ ein Pfad von $s$ nach $t$ in $G_M$.


\section*{Aufgabe 3}
% Zeige O(V*(V+E))
Betrachte den in Aufgabe 2 konstruierten Graphen $G_M$.
Die Konstruktion von $G_M$ ist in $O(|V|*(|V|+|E|))$ realisierbar.
$A'$ und $B'$ werden in konstanter Zeit konstruiert (ggf $O(\log|V|)$, je nach Datenstruktur).
$E'$ wird in $O(|V|*|E|)$ konstruiert, da f\"ur jede Kante in $E$ noch gepr\"uft werden muss, ob sie in $M$ liegt.
Eine Tiefensuche auf $G_M$ dauert $O(|V|+|E|)$.
Der gefundene Pfad hat h\"ochstens L\"ange $O(|V|)$, jede Kante kann h\"ochstens $O(\log(|V|)^2) = O(|V|)$ eingef\"ugt oder entfernt werden, falls die Kanten in Form von Nachbarschaftslisten gespeichert sind (Adjazenzmatrizen sind da irgendwo in $O(1)$).
\begin{equation*}
O(|V|*(|V|+|E|)) + O(|V|+|E|) + O(|V|^2) \subseteq O(|V|*(|V|+|E|))
\end{equation*}

\section*{Aufgabe 4}

    \begin{enumerate}[label={\alph*)}]
        \item Ein gültiges Matching ist eine Menge $M \subseteq E$ wenn keine
            zwei unterschiedliche Elemente aus $M$ einen gemeinsamen Knoten
            haben.
            Ein Matching $M'$ heißt nicht erweiterbar, wenn es keine Kante
            $ e \in E \setminus M$ gibt, sodass $\{e\} \cup M$ ein gültiges
            Matching ist.
        \item
            TODO: ``Pseudocode Carsten...''
        \item
            $\zz$: $M$ ist nicht erweiterbares Matching $\Rightarrow
            |M| \geq \frac{1}{2}|M_{max}|$
            \\
            \textbf{Annahme:} $M$ ist nicht erweiterbares Matching und
            $|M| < \frac{1}{2}|M_{max}|$
            \\
            Dann muss es aber eine Kante $m_0 \in M_{max}$ geben für
            die gilt: $m_0 \notin M$ und die Kante Verbindet zwei Knoten,
            die von keiner Kante in $M$ überdeckt werden
            \\
            $\Rightarrow M$ kann um $m_0$ erweitert werden
            \\
            $\Rightarrow M$ ist nicht ein nicht erweiterbares Matching
            $\lightning$ Widerspruch!
        \item

    \end{enumerate}

\section*{Aufgabe 06}

    \textbf{Beweise den Satz von König:} \\
    Sei $G = (V,E)$ ein bipartiter Graph und $V$ ist unterteilt in
    die Mengen $L$ und $R$. Sei $M$ ein maximales Matching für $G$.
    Da kein Knoten einer Knotenüberdeckung mehr als eine Kante von $M$
    überdecken kann, muss eine Knotenüberdeckung mit $|M|$ Knoten eine
    maximale Knotenüberdeckung sein.
    \\
    \textbf{Zeige, dass eine solche Überdeckung konstruiert werden kann:}
    \\
    Sei $U$ die Menge der ungematchten Knoten in $L$ und sei $Z$ die Menge
    der Knoten die entweder in $U$ selber sind, oder durch alternierende Pfade mit
    Knoten in $U$ verbunden sind. Ferner sei
    $$
        K = (L\setminus Z) \cup (R\cap Z)
    $$
    jede Kante $e$ in $G$ erfüllt eine der folgenden Bedingungen:
    \begin{itemize}
        \item[a)] $e$ ist Teil eines alternierenden Pfades (und hat einen ``rechten''
        Endpunkt in $K$)
        \item[b)] $e$ hat einen ``linken'' Endpunkt in $K$
    \end{itemize}
    Für den Fall, dass $e$ gematched aber nicht in einem alternierenden Pfad ist, gilt:
    dass $e$ nicht in einem alternierenden Pfad sein kann und in $L\setminus Z$
    enthalten ist. Für den Fall, dass $e$ nicht gematched ist und nicht in
    einem alternierenden Pfad ist gilt: der linke Endpunkt  kann ebenfalls nicht
    in einem alternierenden Pfad sein (denn dieser könnte durch $e$ erweitert werden).
   \\
   Ebenso gilt: jeder Knoten $k \in K$ ist ein Endpunkt von einer gematchten
   Kante, da gilt:
   \begin{itemize}
        \item  falls $k \in L\setminus Z $: $k $ist gematched, da $Z$ eine Obermenge von $U$ ist.
        \item falls $k \in R\cap Z$: $k$ ist gematched, da ? [verstehe da gerade Wikipedias Argumentation nicht ganz]

   \end{itemize}
    Letztendlich kann also keine Kante beide Endpunkte in $K$ haben
    $\Rightarrow$ K ist eine Knotenüberdeckungen mit $|K| = |M|$ und $K$ muss
    eine minimale Knotenüberdeckung sein.

\section*{Aufgabe 07}

	$\zz$: für alle $u,v, \in V$ mit $num(u) > num(v)$ gilt:\\
	$\exists$ Pfad $P$ von $v$ nach $u$ mit $num(w) \leq num(v) \forall w$
	auf $P \Rightarrow u$ und $v$ sind in derselben Zusammenhangskomponente
	\\
	\textbf{Beweis:}\\
	Damit $u$ und $v$ in der selben Zusammenghangskomponente sind, muss gelten:
	\begin{enumerate}[label={\alph*.}]
		\item
			$\exists$ Pfad von $u$ nach $v$
		\item
			$\exists$Pfad von $v$ nach $u$
	\end{enumerate}

	Die zweite Bedingung ist durch die Aufgabenstellung erfüllt. Bleibt zu zeigen:
	$\exists$ Pfad von $u$ nach $v$. \textbf{Annahme:} $\nexists$
	Pfad von $u$ nach $v \Rightarrow$ $v$ kann kein Knoten im Teilbaum mit
	Wurzel $u$ (im Tiefensuche Baum) sein. Betrachte den Pfad:
	$$
		P = [v = w_1, \ldots, w_k = u]
	$$
	Zudem gilt: offensichtlich kann die Tiefensuche beim Betreten des Knotens $v$
	den Knoten $u$ noch nicht betrachtet haben (weil $num(u) > num(v)$).
	Betrachte nun den Knoten $w_i$ (evtl. auch $w_1$), der beim Betreten von $v$ schon betrachtet
	wurde und am nächsten zu u liegt (also den Knoten mit maximalem i,für den diese
	Bedingung gilt). Dieses i muss kleiner $k$ sein ($w_K = u$ ist unbetrachtet).
	Dann können folgende Fälle auftreten:
	\begin{itemize}
		\item $w_i$ ist nicht im Tiefensuche-Keller
			$\Rightarrow w_i$ ist fertig abgearbeitet und wurde aus
			dem Stack entfernt \\
			$\Rightarrow $ alle Kindknoten von $w_i$ wurden bereits betrachtet\\
			$\Rightarrow w_{i+1}$ wurde auch schon betrachtet.\\
			$\Rightarrow i$ war nicht maximal gewählt $\lightning$
		\item $w_i$ ist im Tiefensuche Keller \\
			$\Rightarrow u$ muss in einem Teilbaum mit Wurzel $v$ sein (sonst wäre für das maximale $w_i$
				der obere Fall eingetreten)\\
			$\Rightarrow num(v) > num(u) \lightning$
	\end{itemize}
	$\Rightarrow$ Annahme kann nicht zutreffen \hfill $\qed$

\section*{Aufgabe 08}

	\begin{enumerate}[label={\alph*}]

		\item
			\textbf{Definition reduzierter Graph}: \\
			$G'_{red} = (V_{red},E_{red})$ mit
			\begin{itemize}
				\item
					jeder Knoten $v_{i} \in V_{red}$ steht für eine
					Zusammenhangskomponente $G'_i=(V_i,E'_i)$
				\item
					$(v_i,v_j) \in E_{red} \Leftrightarrow \exists u_i \in V_i,
						u_j \in V_j: (u_i,u_j) \in E'$
			\end{itemize}

		\item
		\textbf{Beweis} (aus dem Buch) : \\
		(\textbf{R}) $\Rightarrow$ Für alle Knoten $v$ die während der Tiefensuche
			auf G' mit Startknoten $u$ besucht werden gilt: $\exists$ Pfad P in G von
			$v$ nach $u$ mit $num(w) \leq num(u) \forall w$ auf P. Dann gibt es zwei Fälle:
			\begin{itemize}
				\item
					$num(u) > num(v) \Rightarrow u$ und $v$ sind in der selben Zusammenhangskomponente
					(siehe voherige Aufgabe).
				\item
					$num(u) \leq num(v) \Rightarrow $ nach (\textbf{R}) wurde die Komponente
						die $u$ enthält bereits konstruiert.
			\end{itemize}

	\end{enumerate}
\section*{Aufgabe 12}

    \begin{itemize}
        \item   Einzelpaar-kürzeste-Weg-Problem (single-pair shortest-path problem oder
        single-source single-sink shortest-path problem): Gegeben $G = (V, E, w)$ und zwei
        Knoten $s,t \in V$ möchte man einen kürzesten Pfad von  $s$ nach $t$ finden.
        \item Einzelquelle-kürzeste-Weg-Problem (single-source shortest-path problem):
        Gegeben $G = (V, E, w)$ und einen (Quell-)Knoten $s \in V$, soll für jeden Knoten $v \in V$
        ein kürzester Pfad von $s$ zu $v$ gefunden werden.
        \item Alle-Paare-kürzeste-Weg-Problem (all-pairs shortest-path problem): Hier möchte
		man, gegeben $G = (V, E, w)$, für jedes Paar $u,v \in V$, einen kürzesten Pfad von u nach v
		finden.
	\end{itemize}

\section*{Aufgabe 13}
    \textbf{Annahme: } Sei $P$ ein kürzester Pfad und $\exists P_{ij}$
    wobei $P_{ij}$ nicht der kürzeste Pfad von $v_i$ nach $v_j$ ist.
    Dann folgt trivialerweise, dass es einen kürzeren Pfad $P'_{ij}$ gibt
    und einen Pfad $P' = v_1, \ldots, v_i, \ldots, v_j, \ldots, v_k$ mit
    $|P'| = |P| - (|P_{ij}| - |P'_{ij}|)$ \\ $\Rightarrow |P'| < |P|$
    \\ $\Rightarrow P$ ist nicht kürzester Pfad $\lightning$



\section*{Aufgabe 21}
	\paragraph{Annahme:} $(u,v)$ ist nicht sicher bzgl. $\tilde{A}$, also $(u, v) \notin A$.
	Konstruiere anderen überspannenden Baum $T' = (V, A', w)$ mit $\tilde{A} \cup \{(u,v)\} \subseteq A'$ wie folgt:

	Betrachte Pfad $P$ von $u$ nach $v$ in $T$. Dieser bildet zusammen mit $(u, v)$ einen Kreis.
	Da $(u,v)$ den Schnitt $(S, V \setminus S)$ kreuzt, existiert $(x, y) \in P$ so dass $(x, y)$ den Schnitt $(S, V \setminus S)$ kreuzt.
	Entfernt man $(x,y)$ aus $T$ und fügt $(u,v)$ ein, so erhält man den neuen überspannenden Baum $T'$
	mit $A' = A \setminus \{(x, y)\} \cup \{(u, v)\}$.
	Da $(u, v)$ eine Kandidatskante war, gilt $w(u,v) \leq w(x, y)$.
	\begin{equation*}
	w(T') = w(T) - w(x, y) + w(u,v) \leq w(T)
	\end{equation*}
	Da $T$ bereits ein minimaler überspannender Baum war, muss $w(T') = w(T)$ sein und $T'$ ist ein minimaler überspannender Baum.
	$(x, y) \notin \tilde{A}$, da $\tilde{A}$ den Schnitt $(S, V \setminus S)$ respektiert, $(x, y)$ diesen aber kreuzt.
	Also ist $\tilde{A} \subseteq A'$. $(u, v)$ sicher bzgl. $\tilde{A}$.

\section*{Aufgabe 22}
	\paragraph{Eingabe:} Ungerichteter, gewichteter Graph $G=(V, E, w)$
	\paragraph{Ausgabe:} Minimaler $G$ überspannender Spannwald $T=(V, A, w)$
	\begin{lstlisting}[language=Java]
A = new Set();
Q = new Array(E);
sort(Q, incr by w);
UF = makeUnionFind(V);
for ((u, v) in Q)
{
	idU = UF.find(u);
	idV = UF.find(v);
	if (idU != idV)
	{
		UF.union(idU, idV);
		A.insert((u, v));
	}
}
return T = (V, A, w);
	\end{lstlisting}


\section*{Aufgabe 23}
Sei $G = (V, E, w)$ ungerichteter, gewichteter Graph mit $\forall e_1, e_2 \in E : w(e_1) \neq w(e_2)$, $T = (V, A, w)$ minimaler überspannender Baum von G. $T$ existiert, da $G$ als zusammenhängend vorausgesetzt wird.
\paragraph{Wissen} Jeder minimale überspannende Baum $T$ von $G$ hat genau $|V|-1$ viele Kanten. Gäbe es weniger Kanten, gäbe es einen Knoten, der nicht mit dem Baum verbunden wäre, gäbe es mehr Kanten, so gäbe es einen Kreis.
\paragraph{}
Wähle eine beliebige Kante $(u, v) \in A$ und entferne diese. Damit T nun wieder ein überspannender Baum wird, muss eine geeignete Kante $(x, y) \in E \setminus \{(u,v)\}$ wieder hinzugefügt werden.
Sei $T' = (V, A', w)$ mit $A' = A \setminus \{(u,v)\} \cup \{(x,y)\}$.
Da alle Kanten unterschiedliche Gewichte haben, gilt $w(u,v) \neq w(x, y)$.
\begin{itemize}
	\item Falls $w(u, v) < w(x, y)$, so ist $w(T') > w(T)$ und T' kein minimaler überspannender Baum mehr.
	\item Falls $w(u, v) > w(x, y)$, so ist $w(T') < w(T)$, was ein widerspruch dazu ist, dass $T$ ein minimaler überspannender Baum war.
\end{itemize}

Also existiert genau ein minimaler überspannender Baum $T$ von $G$.

\end{document}
