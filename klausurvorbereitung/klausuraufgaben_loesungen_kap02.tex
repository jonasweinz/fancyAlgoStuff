% header
\documentclass[10pt,a4paper]{article}

\usepackage[utf8]{inputenc}
\usepackage{hyperref}
\usepackage{amssymb}
\usepackage{amsmath}
\usepackage{ngerman}
\usepackage{gensymb}

% the document
\begin{document}

% create the title
\title{Klausuraufgaben \\
\small{Kapitel 8 - Algorithmen auf Graphen}}
\author{Tom Kneiphof}
\date{\today}
\maketitle

\section*{Aufgabe 1}

\paragraph{a)}
Ein Matching $M \subseteq E$ eines ungerichteten Graphens $G = (V, E)$ ist eine Teilmenge der Kanten, so dass kein Knoten an zwei Kanten beteiligt ist.
\begin{equation*}
\forall \{v,w\}, \{x, y\} \in M \subseteq E : \{v, w\} \cap \{x, y\} = \emptyset
\end{equation*}


\paragraph{b)}
Bei einem \textbf{Nicht erweiterbaren Matching} handelt es sich um ein Matching $M$, zu dem keine Kante aus $E$ hinzugenommen werden kann, so dass es ein Matching bleibt.
\begin{equation*}
\forall \{v, w\} \in E \setminus M : M \cup \{ \; \{v, w\} \; \} \; \textnormal{ist kein Matching}
\end{equation*}

Bei einem \textbf{Maximales Matching} handelt es sich um ein Matching $M_{max}$ maximaler Gr\"o{\ss}e.
Es gibt also kein Matching welches mehr Kanten enth\"alt als $M_{max}$. Es gilt:
\begin{equation*}
\forall M : | M | \leq | M_{max} |
\end{equation*}


\paragraph{c)}
Satz von Berge: Ein Matching $M$ ist genau dann maximal, falls es keinen $M$-augmentierenden Pfad gibt.

\paragraph{d)}
Beweis des Satzes von Berge:

\paragraph{$\Rightarrow$}
Annahme: Es existiert ein $M$-augmentierender Pfad $P$.
Definiere $M' := M \oplus P$. Dann gilt $|M'| > |M|$, da $n$ Kanten entfernt wurden und $n+1$ Kanten hinzugef\"ugt wurden.
Dies ist ein Widerspruch dazu, dass $M$ bereits ein Maximales Matching war.

\paragraph{$\Leftarrow$}
Sei $M$ ein Matching in $G$, und existiere kein $M$-augmentierender Pfad. Sei weiter $M_{max}$ ein beliebiges maximales Matching in $G$. Annahme: $|M| < |M_{max}|$.
Betrachte $G' := (V, M \oplus M_{max})$. Da aus beiden Matchings je h\"ochstens eine Kante inzident zu $v$ sein kann, gilt:
\begin{equation*}
\forall v \in V : deg_{G'}(v) \leq 2
\end{equation*}
Die Zusammenhangskomponenten von $G'$ sind also isolierte Knoten, Kreise gerader L\"ange mit abwechselnden Kanten aus $M$ und $M_{max}$ oder Pfade mit abwechselnden Kanten aus $M$ und $M_{max}$.
Da $|M| < |M_{max}|$ angenommen wurde, gibt es einen $M$-augmentierenden Pfad $P$ in $G'$, dessen Endknoten beide $M$-frei sind. Da $P$ auch in $G$ existiert, gibt es einen $M$-augmentierenden Pfad in $G$, was ein Widerspruch dazu ist, dass ein solcher Pfad nicht existiert.

\end{document}
